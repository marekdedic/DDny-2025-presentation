\documentclass[10pt, aspectratio=169]{beamer}
\setbeamertemplate{navigation symbols}{}

\usepackage{polyglossia}

\usepackage{adjustbox}
\usepackage{amsfonts}
\usepackage{amsmath}
\usepackage{amsthm}
\usepackage{appendixnumberbeamer}
\usepackage[backend=biber,style=iso-authoryear,sortlocale=en_US,autolang=other,bibencoding=UTF8]{biblatex}
\usepackage{booktabs}
\usepackage{color}
\usepackage{csquotes}
\usepackage{datetime}
\usepackage{fontspec}
\usepackage{graphics}
\usepackage{hyperref}
\usepackage{makecell}
\usepackage{mathrsfs}
\usepackage{microtype}
\usepackage{multirow}
\usepackage{pifont}
\usepackage[mode=buildnew]{standalone}
\usepackage{subcaption}
\usepackage{tabularx}
\usepackage{tikz}
\usepackage{todonotes}
\usepackage{ulem}
\usepackage{url}

\addbibresource{zotero.bib}

\setdefaultlanguage{english}
\setotherlanguage{czech}
\setmainfont{TeX Gyre Termes}
\usetheme{Boadilla}
\usecolortheme{crane}
\setbeamertemplate{title page}[default][rounded=true,shadow=false]
\setbeamertemplate{section in toc}[ball unnumbered]
\setbeamertemplate{bibliography item}{}

\hypersetup{%
	pdfencoding=auto,
	unicode=true,
	citecolor=green,
	filecolor=blue,
	linkcolor=red,
	urlcolor=blue
}

\makeatletter
\newcommand*{\currentSection}{\@currentlabelname}
\makeatother

\def\mathdefault#1{#1}

\newcommand{\name}[1]{\textit{#1}}
\newcommand{\mathfield}{\ensuremath{\mathbb}}
\newcommand{\mathmat}{\ensuremath{\mathbf}}
\newcommand{\mathset}{\ensuremath{\mathbb}}
\newcommand{\mathspace}{\ensuremath{\mathcal}}
\newcommand{\mathvec}{\ensuremath{\bm}}

\DeclareMathOperator*{\argmin}{arg\,min}
\DeclareMathOperator*{\argmax}{arg\,max}


\title[HPO on for GNNs]
{%
	Benchmarking and Transfer Learning for Hyperparameter Optimization of Graph Neural Networks
}

\newdate{presentation}{25}{11}{2025}
\date[November 2025]{FNSPE Doctoral days, \displaydate{presentation}}

\author[Marek Dědič]
{%
	Marek~Dědič\inst{1, 2}
}

\institute[CTU in Prague, Cisco]
{%
	\inst{1} Czech Technical University in Prague \and
	\inst{2} Cisco Systems, Inc.
}

\begin{document}

\begin{frame}
	\titlepage
\end{frame}

\begin{frame}{Parameters \& hyperparameters}
	\begin{itemize}
		\item Learners have parameters and hyperparameters
		\item Parameters optimized automatically
		\item Hyper-parameters set by humans
	\end{itemize}
	\centering
	\includegraphics[width=0.5\linewidth]{images/HPO.png}\footcite{kaiku_hyperparameter_2023}
\end{frame}

\begin{frame}{GNN hyperparameter setups}
	\centering
	\only<1>{\includestandalone[width=\linewidth, page = 1]{images/hyperparameter-setups/hyperparameter-setups}}
	\only<2>{\includestandalone[width=\linewidth, page = 2]{images/hyperparameter-setups/hyperparameter-setups}}
	\only<3>{\includestandalone[width=\linewidth, page = 3]{images/hyperparameter-setups/hyperparameter-setups}}
\end{frame}

\begin{frame}{Hyper-parameter optimization}
	\begin{itemize}
		\item Hyper-parameters need to be found (semi-)automatically
		\item No gradients \( \rightarrow \) black-box problem
		\item \textit{Expensive} black-box problem
	\end{itemize}
\end{frame}

\begin{frame}{HPO loop}
	\centering
	\includegraphics[width=0.8\linewidth]{images/HPO-loop.pdf}\footcite{bischl_hyperparameter_2023}
\end{frame}

\begin{frame}{HPO formalization}
	Formally, a HPO method \( \tau \) is a function that proposes a hyperparameter configuration \( \hat{\lambda} \) to try next:
	\[
		\hat{\lambda} = \tau \left( \mathcal{D}, \mathscr{F}, \tilde{\Lambda}, \rho \right)
	\]
	given
	\begin{itemize}
		\item \( \mathcal{D} \): dataset
		\item \( \mathscr{F} \): learning algorithm
		\item \( \tilde{\Lambda} \): search space
		\item \( \rho \): performance metric
	\end{itemize}
\end{frame}

\begin{frame}{Contributions}
	\textbf{Goals \& contributions}:
	\begin{itemize}
		\item Systematic HPO Benchmark on GNNs
		\item Meta-Learning for Hyperparameter Transfer\footnote{Building on previous work \cite{prochazka_which_2023}}
	\end{itemize}
\end{frame}

\begin{frame}{Evaluated standard algorithms}
	\begin{itemize}
		\item Grid search
		\item Random search
		\item Bayesian optimization
		\item Sobol Quasi-Monte Carlo
		\item Tree-structured Parzen Estimator
	\end{itemize}
\end{frame}

\begin{frame}{Experiment setup}
	\begin{itemize}
		\item 9 standard datasets (34 nodes -- 169k nodes)
		\item GraphSAGE model with 8 hyperparameters
		\item 10 repeated iterations for each configuration
		\item HPO early-stopping based on wall-clock time \& improvement
	\end{itemize}
\end{frame}

\begin{frame}{Benchmark results}
	\begin{columns}[t,onlytextwidth]
		\begin{column}{0.45\textwidth}
			\begin{itemize}
				\item SMBO methods (BO and TPE) outperform other methods across all datasets. Between BO and TPE there is no clear winner.
				\item Simple methods like random search and grid search remain solid choices for HPO as they usually trail behind SMBO methods by only a small margin.
				\item Sobol QMC consistently ranks worse than SMBO methods.
			\end{itemize}
		\end{column}

		\begin{column}{0.52\textwidth}
			\centering
			\scriptsize
			\begin{table}
				\centering
				\begin{tabular}{lccccc}
					\toprule
					\textbf{Dataset} & \textbf{Random}    & \textbf{Grid} & \textbf{BO}        & \textbf{TPE}       & \textbf{QMC}       \\
					\midrule
					Cora             & 0.8560             & 0.8491        & \textbf{0.8747}    & \underline{0.8659} & 0.8351             \\
					CiteSeer         & 0.7146             & 0.7046        & \underline{0.7172} & \textbf{0.7236}    & 0.7089             \\
					Squirrel         & \underline{0.3659} & 0.3605        & 0.3544             & \textbf{0.3755}    & 0.3549             \\
					PubMed           & 0.8515             & 0.8457        & \textbf{0.8825}    & \underline{0.8643} & 0.8596             \\
					CoraFull         & 0.6211             & 0.6371        & \underline{0.6450} & \textbf{0.6555}    & 0.6385             \\
					DBLP             & 0.8051             & 0.7996        & \textbf{0.8118}    & 0.8085             & \underline{0.8088} \\
					Computers        & 0.6973             & 0.5999        & \textbf{0.8945}    & \underline{0.8047} & 0.7428             \\
					Flickr           & 0.0864             & 0.0961        & \textbf{0.1908}    & \underline{0.1460} & 0.1069             \\
					ArXiv            & 0.3950             & 0.3796        & \underline{0.3987} & \textbf{0.4098}    & 0.3955             \\
					\bottomrule
				\end{tabular}
				\caption{Final F1 scores for each HPO method and for each dataset, averaged over 10 independent runs. The best method is \textbf{bold} and the second best is \underline{underlined}.}
			\end{table}
		\end{column}
	\end{columns}
\end{frame}

\begin{frame}{Benchmark results -- ranks}
	\begin{center}
		\adjustbox{width=0.8\columnwidth}{%
			\input{images/benchmark_ranks/benchmark_ranks.pgf}
		}
	\end{center}
	Ranks-over-time of the benchmarked HPO methods over the progress of the optimization, aggreagated over 10 independent runs for each of 9 datasets.
\end{frame}

\begin{frame}{An RF-based approach}
	We assume that we have:
	\begin{itemize}
		\item A graph property extraction function \( \delta : \mathrm{D} \to \mathfield{R}^d \)
		\item A meta-model \( \mathcal{M}_\rho : \mathfield{R}^d \times \boldsymbol\Lambda \to \mathfield{R} \) that predicts a performance metric \( \rho \) given dataset properties and hyperparameter configuration
	\end{itemize}
	We propose a HPO method using such a metamodel:
	\begin{equation*}
		\tau \left( \mathcal{D}, \mathscr{F}, \tilde{\Lambda}, \rho \right) = \argmax_{\lambda \in \tilde{\Lambda}} \mathcal{M}_\rho \left( \delta \left( \mathcal{D} \right), \lambda \right)
	\end{equation*}
	Meta model is cheap \( \rightarrow \) \( \argmax \) is feasible
\end{frame}

\begin{frame}{Meta-dataset}
	A meta-dataset used to train the metamodel:
	\begin{center}
		\resizebox{\linewidth}{!}{%
			\begin{tabular}{rrrrrrrr}
				\toprule
				\multicolumn{4}{c}{\textbf{Dataset properties}} & \multicolumn{3}{c}{\textbf{Hyper-parameters}} & \textbf{Perf.\ metrics} \\
				\cmidrule(r){1-4} \cmidrule(lr){5-7} \cmidrule(l){8-8}
				\textbf{Node count} & \textbf{Class ratio} & \textbf{\# of components} & \textbf{Avg.\ node degree} & \textbf{\# of layers} & \textbf{Dropout} & \textbf{Activation} & \textbf{F1} \\
				\midrule
				6 & 1 / 3 & 1 & 2 & 2 & 0.3 & ReLU & 0.65 \\
				6 & 1 / 3 & 1 & 2 & 3 & 0.3 & ReLU & 0.7 \\
				6 & 1 / 3 & 1 & 2 & 2 & 0.5 & ReLU & 0.62 \\
				6 & 1 / 3 & 1 & 2 & 3 & 0.5 & ReLU & 0.6 \\
				\bottomrule
			\end{tabular}
		}
	\end{center}
\end{frame}

\begin{frame}{The Cross-RF metamodel}
	Can we pretrain the metamodel on other datasets and transfer to a new one?

	\vspace{1cm}
	\hspace{1cm}\uncover<2->{\textbf{Yes.}}
\end{frame}

\begin{frame}{Experiment setup}
	\begin{itemize}
		\item Used a random forrest regressor as the meta-model
		\item 100 trees, default paramers from \texttt{scikit-learn}
		\item For each of the 9 datasets, pretrain Cross-RF on the other 8
		\item In each step, all of the already evaluated configurations are excluded from \( \tilde{\Lambda} \)
		\item In all experiments, we used the F1-score as the performance metric \( \rho \)
		\item Compared with BO and TPE (best methods from the benchmark)
	\end{itemize}
\end{frame}

\begin{frame}{Meta-model results}
	\begin{columns}[t,onlytextwidth]
		\begin{column}{0.45\textwidth}
			\begin{itemize}
				\item Cross-RF achieves the best final performance on 4 out of the 9 datasets, while being second best on another 3 datasets.
				\item Only on the CiteSeer and CoraFull datasets does Cross-RF perform worse than both reference methods.
				\item Overall, Cross-RF achieves an average rank of 1.78, outperforming both BO (2.0) and TPE (2.22).
			\end{itemize}
		\end{column}

		\begin{column}{0.52\textwidth}
			\begin{table}
				\centering
			\scriptsize
				\begin{tabular}{lccccc}
					\toprule
					\textbf{Dataset} & \textbf{BO}        & \textbf{TPE}       & \textbf{Cross-RF}  \\
					\midrule
					Cora             & \textbf{0.8747}    & 0.8659             & \underline{0.8727} \\
					CiteSeer         & \underline{0.7172} & \textbf{0.7236}    & 0.7095             \\
					Squirrel         & 0.3544             & \underline{0.3755} & \textbf{0.3769}    \\
					PubMed           & \textbf{0.8825}    & 0.8643             & \underline{0.8807} \\
					CoraFull         & \underline{0.6450} & \textbf{0.6555}    & 0.6426             \\
					DBLP             & \underline{0.8118} & 0.8085             & \textbf{0.8123}    \\
					Computers        & \underline{0.8945} & 0.8047             & \textbf{0.9023}    \\
					Flickr           & \underline{0.1908} & 0.1460             & \textbf{0.1956}    \\
					ArXiv            & 0.3987             & \textbf{0.4098}    & \underline{0.4094} \\
					\bottomrule
				\end{tabular}
				\caption{Final F1 scores for each HPO method and for each dataset, averaged over 10 independent runs. The best method is \textbf{bold} and the second best is \underline{underlined}.}
			\end{table}
		\end{column}
	\end{columns}
\end{frame}

\begin{frame}{Meta-model results -- ranks}
	\begin{center}
		\adjustbox{width=0.8\columnwidth}{%
			\input{images/cross-rf_ranks/cross-rf_ranks.pgf}
		}
	\end{center}
	Ranks-over-time of Cross-RF and reference methods over the progress of the optimization, aggreagated over 10 independent runs for each of 9 datasets.
\end{frame}

\begin{frame}{Future work}
	Immediate next steps:
	\begin{itemize}
		\item Including a broader range of GNN architectures
		\item Considering other performance metrics
		\item Comparing to newer HPO methods
	\end{itemize}

	Longer-term directions:
	\begin{itemize}
		\item Pre-training the metamodel on sysnthetic datasets generated from known graph models
		\item Using a different metamodel architecture which is better suited to continuous search spaces
		\item Incorporating the metamodel as a surrogate model within an SMBO framework
	\end{itemize}
\end{frame}

\begin{frame}{Conclusion}
	\begin{itemize}
		\item Evaluated standard HPO algorithms for GNNs
		\item Shown the metamodel hyperparameter optimization algorithm for GNNs to beat standard methods
		\item Implemented as a standard Optuna sampler
		\item More interesting research to come
	\end{itemize}
\end{frame}

\begin{frame}
	\titlepage
\end{frame}

\appendix

\begin{frame}[allowframebreaks]{Bibliography}
	\printbibliography
\end{frame}

\begin{frame}
	\begin{table}[ht]
		\begin{tabular}{lrrrr}
			\toprule
			\textbf{Dataset} & \textbf{\#Nodes} & \textbf{\#Edges} & \textbf{\#Features} & \textbf{\#Classes} \\
			\midrule
			Cora		     & 2\,708           & 10\,556          & 1\,433              & 7                  \\
			CiteSeer         & 3\,327           & 9\,104           & 3\,703              & 6                  \\
			Squirrel         & 5\,201           & 198\,493         & 3\,169              & 5                  \\
			PubMed           & 19\,717          & 88\,648          & 500                 & 3                  \\
			CoraFull         & 19\,793          & 126\,842         & 8\,710              & 70                 \\
			DBLP             & 17\,716          & 105\,734         & 1\,639              & 4                  \\
			Computers        & 13\,752          & 491\,722         & 767                 & 10                 \\
			Flickr           & 89\,250          & 899\,756         & 500                 & 7                  \\
			ArXiv            & 169\,343         & 1\,166\,243      & 128                 & 40                 \\
			\bottomrule
		\end{tabular}
		\caption{Datasets used in the benchmark}
	\end{table}
\end{frame}

\begin{frame}
	\begin{table}
		\resizebox{\linewidth}{!}{%
			\centering
			\begin{tabular}{lcccccccccc}
				\toprule
				\textbf{Dataset} & \textbf{Ep.} & \textbf{Lyrs.} & \textbf{HW} & \textbf{Act.}                                & \textbf{Opt.}                                & \textbf{Drop.}                          & \textbf{Agg.}                              & \textbf{ES} & \textbf{L2}                                       & \textbf{LR}                                          \\
				\midrule
				\multicolumn{11}{l}{\textit{Small datasets}} \\
				\addlinespace
				Cora             & 200          & 1--3           & 16--64      & \multirow{11}{*}{\shortstack{ReLU, \\ PReLU}} & \multirow{11}{*}{\shortstack{Adam, \\ AdamW}} & \multirow{11}{*}{\shortstack{0, \\ 0.5}} & \multirow{11}{*}{\shortstack{Mean, \\ Max}} & 10          & 0, \( 5 \times 10^{-4} \), \( 5 \times 10^{-3} \) & \( 10^{-2} \), \( 5 \times 10^{-3} \), \( 10^{-3} \) \\
				CiteSeer         & 200          & 1--3           & 16--64      &                                              &                                              &                                         &                                            & 10          & 0, \( 5 \times 10^{-4} \), \( 5 \times 10^{-3} \) & \( 10^{-2} \), \( 5 \times 10^{-3} \), \( 10^{-3} \) \\
				Squirrel         & 500          & 1--3           & 32--128     &                                              &                                              &                                         &                                            & 20          & 0, \( 5 \times 10^{-4} \), \( 5 \times 10^{-3} \) & \( 10^{-2} \), \( 5 \times 10^{-3} \), \( 10^{-3} \) \\
				\addlinespace
				\multicolumn{11}{l}{\textit{Medium datasets}} \\
				\addlinespace
				PubMed           & 200          & 1--3           & 16--64      &                                              &                                              &                                         &                                            & 10          & 0, \( 5 \times 10^{-4} \), \( 5 \times 10^{-3} \) & \( 10^{-2} \), \( 5 \times 10^{-3} \), \( 10^{-3} \) \\
				CoraFull         & 500          & 2--3           & 64--128     &                                              &                                              &                                         &                                            & 20          & 0, \( 5 \times 10^{-5} \), \( 5 \times 10^{-4} \) & \( 10^{-2} \), \( 5 \times 10^{-3} \), \( 10^{-3} \) \\
				DBLP             & 500          & 2--3           & 64--128     &                                              &                                              &                                         &                                            & 20          & 0, \( 5 \times 10^{-5} \), \( 5 \times 10^{-4} \) & \( 10^{-2} \), \( 5 \times 10^{-3} \), \( 10^{-3} \) \\
				Computers        & 500          & 2--3           & 64--128     &                                              &                                              &                                         &                                            & 20          & 0, \( 5 \times 10^{-5} \), \( 5 \times 10^{-4} \) & \( 10^{-2} \), \( 5 \times 10^{-3} \), \( 10^{-3} \) \\
				Flickr           & 100          & 2--3           & 128--256    &                                              &                                              &                                         &                                            & 10          & 0, \( 5 \times 10^{-4} \)                         & \( 5 \times 10^{-3} \), \( 10^{-3} \)                \\
				\addlinespace
				\multicolumn{11}{l}{\textit{Large datasets}} \\
				\addlinespace
				ArXiv            & 150          & 2--3           & 256--512    &                                              &                                              &                                         &                                            & 10          & 0, \( 5 \times 10^{-4} \)                         & \( 10^{-2} \), \( 5 \times 10^{-3} \)                \\
				\bottomrule
			\end{tabular}
		}
		\caption{Hyperparameters based on dataset sizes.}
		\vspace{0.05cm}

		{\footnotesize \textbf{Abbreviations}: \textbf{Ep.}: Epochs, \textbf{Lyrs.}: Number of layers, \textbf{HW}: Hidden Layer Width, \textbf{Act.}: Activation Fn., \textbf{Opt.}: Optimizer, \textbf{Drop.}: Dropout Rate, \textbf{Agg.}: Aggregation Fn., \textbf{ES}: Early Stopping patience, \textbf{L2}: \( L_2 \) Regularization, \textbf{LR}: Learning Rate.}
	\end{table}
\end{frame}

\begin{frame}
	\begin{table}
		\resizebox{1\linewidth}{!}{%
			\centering
			\begin{tabularx}{1.75\linewidth}{lcccX}
				\toprule
				\multirow{2}{*}{\textbf{Graph Property}}           & \multicolumn{3}{c}{\textbf{Relates to}}                   & \multirow{2}{*}{\textbf{Description / Definition}} \\
				\cmidrule(lr){2-4}
																   & \textbf{Task} & \textbf{Struct.} & \textbf{Attrs.} & \\
				\midrule
				\textbf{Node count}                                & \crossmark    & \crossmark         & \crossmark          & Number of nodes. \\
				\textbf{Edge count}                                & \crossmark    & \crossmark         & \crossmark          & Number of edges. \\
				\textbf{Number of components}                      & \crossmark    & \checkmark         & \crossmark          & Number of connected components of the graph. \\
				\textbf{Average node degree}                       & \crossmark    & \checkmark         & \crossmark          & Average node degree in the graph. \\
				\textbf{Global assortativity}                      & \crossmark    & \checkmark         & \crossmark          & Measure of the tendency of nodes to connect with other similar nodes, rather than dissimilar nodes. \\
				\textbf{Attribute similarity}                      & \crossmark    & \checkmark         & \checkmark          & Average cosine similarity of attributes across all edges in the graph. \\
				\textbf{Attribute homophily}                       & \checkmark    & \checkmark         & \crossmark          & Measure of how clustered together are nodes with similar attributes. \\
				\textbf{Edge homophily}                            & \checkmark    & \checkmark         & \crossmark          & Fraction of edges connecting nodes of the same class. \\
				\textbf{Node homophily}                            & \checkmark    & \checkmark         & \crossmark          & Fraction of node neighbours having the same class as the node in question, averaged over all nodes. \\
				\textbf{Class homophily}                           & \checkmark    & \checkmark         & \crossmark          & A modification of node homophily that is invariant to the number of classes. \\
				\textbf{Ratio of positive nodes of degree $>1$}    & \checkmark    & \checkmark         & \crossmark          & Ratio of positive nodes with degree greater than one. \\
				\textbf{Fraction of positive nodes of degree $>2$} & \checkmark    & \checkmark         & \crossmark          & The fraction of positive nodes with degree greater than two, out of those with degree greater than one. \\
				\textbf{Average positive node degree}              & \checkmark    & \checkmark         & \crossmark          & Average node degree in the sub-graph restricted to positive nodes. \\
				\textbf{Relative presence of positive edges}       & \checkmark    & \checkmark         & \crossmark          & Number of edges connecting positive nodes, divided by the number of edges that would be present in a theoretical clique constructed of all positive nodes. \\
				\textbf{Positive attribute similarity}             & \checkmark    & \crossmark         & \checkmark          & Avg.\ feature similarity within positive class. \\
				\textbf{Positive to negative attribute similarity} & \checkmark    & \crossmark         & \checkmark          & Avg.\ feature similarity of positive vs negative nodes. \\
				\textbf{Negative to positive attribute similarity} & \checkmark    & \crossmark         & \checkmark          & Avg.\ feature similarity of negative vs positive nodes. \\
				\textbf{Adjusted homophily}                        & \checkmark    & \checkmark         & \crossmark          & Homophily adjusted for degree–class distribution. \\
				\textbf{Node label informativeness}                & \checkmark    & \checkmark         & \crossmark          & Mutual‐information‐based measure of how informative node labels are about the graph’s connectivity. \\
				\textbf{Balanced accuracy}                         & \checkmark    & \checkmark         & \crossmark          & Class‐averaged fraction of edges connecting nodes to same‐class neighbors. \\
				\textbf{Adjusted accuracy}                         & \checkmark    & \checkmark         & \crossmark          & Accuracy metric adjusted to range $[0,1]$ based on balanced accuracy. \\
				\bottomrule
			\end{tabularx}
		}
		\caption{Graph dataset properties considered.}
	\end{table}
\end{frame}

\end{document}
