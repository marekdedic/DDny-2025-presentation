\documentclass[10pt, aspectratio=169]{beamer}
\setbeamertemplate{navigation symbols}{}

\usepackage{polyglossia}

\usepackage{adjustbox}
\usepackage{amsfonts}
\usepackage{amsmath}
\usepackage{amsthm}
\usepackage{appendixnumberbeamer}
%\usepackage[backend=biber,style=iso-authoryear,sortlocale=en_US,autolang=other,bibencoding=UTF8]{biblatex}
\usepackage{booktabs}
\usepackage{color}
\usepackage{csquotes}
\usepackage{datetime}
\usepackage{fontspec}
\usepackage{graphics}
\usepackage{hyperref}
\usepackage{mathrsfs}
\usepackage{microtype}
\usepackage{pifont}
\usepackage{subcaption}
\usepackage{tikz}
\usepackage{todonotes}
\usepackage{ulem}
\usepackage{url}

\setdefaultlanguage{english}
\setotherlanguage{czech}
\setmainfont{TeX Gyre Termes}
\usetheme{Boadilla}
\usecolortheme{crane}
\setbeamertemplate{title page}[default][rounded=true,shadow=false]
\setbeamertemplate{section in toc}[ball unnumbered]
\setbeamertemplate{bibliography item}{}

\hypersetup{%
	pdfencoding=auto,
	unicode=true,
	citecolor=green,
	filecolor=blue,
	linkcolor=red,
	urlcolor=blue
}

\makeatletter
\newcommand*{\currentSection}{\@currentlabelname}
\makeatother

\def\mathdefault#1{#1}

\newcommand{\name}[1]{\textit{#1}}
\newcommand{\mathfield}{\ensuremath{\mathbb}}
\newcommand{\mathmat}{\ensuremath{\mathbf}}
\newcommand{\mathset}{\ensuremath{\mathbb}}
\newcommand{\mathspace}{\ensuremath{\mathcal}}
\newcommand{\mathvec}{\ensuremath{\bm}}

\DeclareMathOperator*{\argmin}{arg\,min}
\DeclareMathOperator*{\argmax}{arg\,max}


\title[RL for structured data]
{%
	Representation learning for structured data
}

\newdate{presentation}{25}{11}{2025}
\date[October 2025]{FNSPE Doctoral days, \displaydate{presentation}}

\author[Marek Dědič]
{%
	Marek~Dědič\inst{1, 2}
}

\institute[CTU in Prague, Cisco]
{%
	\inst{1} Czech Technical University in Prague \and
	\inst{2} Cisco Systems, Inc.
}

\begin{document}

\begin{frame}
	\titlepage
\end{frame}

\begin{frame}{Hyperparameter optimization for GNNs}
	\begin{itemize}
		\item Measure graph properties
		\item Train GNN with different hyper-parameters
		\item Train a meta-model predicting GNN performance from graph properties and hyper-parameters
		\item Use the meta-model to suggest hyper-parameters for new graphs
	\end{itemize}

	\begin{center}
		\includegraphics[width=0.6\linewidth]{images/hyperpar_tuning_single.pdf}
	\end{center}
\end{frame}

\begin{frame}{Hyperparameter optimization for GNNs}
	We benchmarked 5 standard HPO methods on 9 GNN datasets. The results show that Bayesian optimization (BO) and Tree-structured Parzen Estimators (TPE) achieve the best performance, while Quasi Monte-Carlo (QMC) falls behind.
	\adjustbox{width=\columnwidth}{%
		\input{images/benchmark_ranks/benchmark_ranks.pgf}
	}
\end{frame}

\begin{frame}{The Cross-RF method for HPO on graphs}
	We propose a HPO method using a metamodel \( \mathcal{M}_\rho : \mathfield{R}^d \times \boldsymbol\Lambda \to \mathfield{R} \) that takes as input the properties of a graph dataset \( \delta \left( \mathcal{D} \right) \) and a hyperparameter configuration \( \lambda \), and outputs an estimate of a performance metric \( \rho \). This gives us the following HPO procedure:
	\begin{equation*}
		\tau \left( \mathcal{D}, \mathscr{F}, \tilde{\Lambda}, \rho \right) = \argmax_{\lambda \in \tilde{\Lambda}} \mathcal{M}_\rho \left( \delta \left( \mathcal{D} \right), \lambda \right)
	\end{equation*}
	We show this method to outperform all of the above methods on the 9 graph datasets by a slight margin.
\end{frame}

\end{document}
